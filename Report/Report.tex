\documentclass{article}
\usepackage{comment}
\usepackage{titling}
\usepackage{pgfplots}
\usepackage{filecontents}
\usepackage[a4paper, total={6in, 8in}]{geometry}

\title{Clustering for Graph Partitioning}
\author{Andrea Smith}

\begin{document}
\begin{titlingpage}
\setlength{\droptitle}{30pt}
\maketitle
\begin{center}
{\fontfamily{lmtt}\selectfont
\large\centerline{About the Author}
}
The author has earned her Bachelor's of Science degree in Computer Science from Missouri University of Science and Technology. She is working towards her doctorate degree in Computer Science, with a focus on graph data mining, from the same university, anticipating completing 2022.
\end{center}
\end{titlingpage}

\tableofcontents
\newpage

\section{Executive Summary}
\section{Introduction}
Stuff to say \cite{kernel,distK}.
\section{Project Specifications}
\section{Detailed Design}
\section{Experimental Results}


\newpage
\section{Appendix A}


\newpage
\begin{thebibliography}{9}

\bibitem{kernel}
I. Dhillon, Y. Guan, and B. Kulis, ``A Fast Kernel-based Multilevel Algorithm for Graph Clustering,'' \textit{KDD}. Chicago, Ill.: 2005.
 
\bibitem{distK} 
J. Edachery, A. Sen, and F. Brandenburg, ``Graph Clustering Using Distance-k Cliques,'' Arizona State University, 1999.
 
\bibitem{kmeans} 
K. Nazeer and M. Sebastian, ``Improving the Accuracy and Efficiency of the k-means Clustering Algorithm,'' 
\textit{Proceedings of the World Congress on Engineering}. London, U.K.: 2009.

\end{thebibliography}

\begin{comment}
\begin{figure}[!htb]
\begin{center}
	\includegraphics[height=20em]{puppy.png}
	\caption{T-Test for Relative Fitness with Parsimony Pressures 0.2 and 0.3}
	\label{tTestMLR}
\end{center}
\end{figure}
\end{comment}

\end{document}